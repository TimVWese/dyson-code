% Dyson-code: Tutorial and Reproducibility Notes
\documentclass[11pt]{article}
% Basic preamble for the Dyson SIS tutorial
\usepackage[a4paper,margin=1in]{geometry}
\usepackage[T1]{fontenc}
\usepackage[utf8]{inputenc}
\usepackage{lmodern}
\usepackage{hyperref}
\usepackage{xcolor}
\usepackage{graphicx}
\usepackage{caption}
\usepackage{amsmath,amssymb}
\usepackage{listings}
\usepackage{fancyvrb}

% Graphics
\graphicspath{{../results/figures/}}
\newcommand{\maybeinclude}[2][]{% optional args + path
  \IfFileExists{../results/figures/#2}{\includegraphics[#1]{#2}}{\fbox{Missing figure: \texttt{\detokenize{#2}}}}%
}


% Convenience
\newcommand{\code}[1]{\texttt{#1}}
\newcommand{\pkg}[1]{\textsf{#1}}

\title{Dyson SIS Code Tutorial and Reproducibility Notes}
\author{Tim Van Wesemael, Gilberto Nakamura}
\date{September 2025}

\begin{document}
\maketitle
\tableofcontents
\bigskip

\newpage

\section{Overview}
This document accompanies the \emph{dyson-code} repository and explains how to reproduce the figures and explore some model variations.
The code implements a Susceptible-Infected-Susceptible (SIS) model on a graph and computes a Dyson similarity transformation that approximately symmetrizes the generator matrix.
The primary entry points are in \code{src/dyson.jl} and runnable scripts in \code{scripts/} that generate the paper figures into \code{results/figures/}.

\paragraph{Key components}
\begin{itemize}
  \item \code{default\_initialisation()} prepares a default complete graph with 6 nodes, default rates, and either loads or computes the Dyson transform.
  \item \code{get\_SIS\_H(A, beta, gamma; incl\_disease\_free)} builds the SIS generator given adjacency matrix \code{A} and rates.
  \item \code{find\_hermitian(H; tol, kmax)} iteratively constructs \(h,\eta\) with reduced asymmetry.
  \item \code{simulate(H, eta; symmetrize, tspan, saveat)} integrates \( x'(t) = -H'x \) with opti-onal similarity transform.
  \item Multiple functions to provide observables and statistics of the system.
\end{itemize}

\section{Quick start}
\subsection{Prerequisites}
\begin{itemize}
  \item Julia 1.11.7 with the packages pinned by \code{Project.toml}
\end{itemize}

\subsection{Reproducing all figures}
From the repository root:
\begin{Verbatim}[fontsize=\small]
$ bash run_all.sh
\end{Verbatim}
This runs every script in \code{scripts/} and writes figures to \code{results/figures/}.

\section{Core API and Usage}
\subsection{API Overview}
The core functionality lives in \code{src/dyson.jl}.
Below is a narrative overview of the exported functions used throughout the scripts, how to apply them in your own work, and when to prefer each.
For a technical summary of arguments and return values, refer to the in-code documentation (docstrings) in \code{src/dyson.jl}.

\paragraph{\code{default\_initialisation()}} Loads cached defaults or generates them if missing: a complete graph with six nodes, SIS generator \code{H}, Hermitized matrix \code{h}, transformation \(\eta\), rates \(\beta,\gamma\), a flag \code{incl\_disease\_free}, and the convergence \code{errors}.
Use this at the top of a script to get a reproducible baseline without waiting for recomputation. Artifacts are cached under \code{results/intermediate/}.
If you want to repeat the experiments for an other system (dynamics, or network), this function serves as a great jumping-off point.
\paragraph{\code{disease\_free\_initialisation()}} Alternative initializer identical to \code{default\_initialisation()} but with the disease-free state included (\code{incl\_disease\_free=true}).
This is useful for exploring regimes where the absorbing disease-free state meaningfully affects statistics or when you want to visualize transitions that cross the absorbing state.
It caches its own intermediate file under a different name (\code{full-6-wdf}). Scripts such as \code{scripts/transition.jl} can adopt this initializer to produce \code{...\_transition.pdf} for the disease-free-inclusive model.


\begin{Verbatim}[fontsize=\small]
name, g, A, H, h, eta, beta, gamma, incl_disease_free, errors = default_initialisation()
\end{Verbatim}

\paragraph{\code{get\_SIS\_H(A, beta, gamma; incl\_disease\_free=false)}} Constructs the SIS generator consistent with the chosen graph (represented by adjacency matrix \code{A}) and rates.
Rows sum to zero by construction, and the disease-free state can be excluded, such that only a single steady-state exists.
Use this when sweeping \(\beta/\gamma\) (see \code{scripts/transition.jl}) or when you want to try a different graph.

\begin{Verbatim}[fontsize=\small]
H = get_SIS_H(A, beta, gamma; incl_disease_free=false)
\end{Verbatim}

\paragraph{\code{find\_hermitian(H; tol, kmax, verbose=false)}} Iteratively finds a similarity transform that reduces the asymmetry of \code{H}, returning the transformed matrix \code{h}, the transform \(\eta\), and an error history.
Use when you need a near-Hermitian representation for analysis or optimization (cf. \code{scripts/steady\_state.jl}).

\begin{Verbatim}[fontsize=\small]
h, eta, errors = find_hermitian(Matrix(H); kmax=5_000)
\end{Verbatim}

\paragraph{\code{SIS\_initial\_state(n)}} Provides the default initial state used in simulations: one infected individual mapped by \(\eta\).
Override this if you need a different starting distribution (e.g., multiple infections or a distribution over states).

\begin{Verbatim}[fontsize=\small]
u0 = SIS_initial_state(size(H, 1))
\end{Verbatim}

\paragraph{\code{simulate(H, eta; symmetrize, tspan, saveat, initial\_func)}} Integrates the dynamics \(x'(t) = -H' x\) under the similarity transform \(\eta\).
Set \code{symmetrize=true} to simulate the symmetrized transformed matrix.
This changes the system depending on how non-Hermitian \code{H} is.
However it ensures that the dynamics are reversible.
Supply a custom \code{initial\_condition} to explore different initial states.

\begin{Verbatim}[fontsize=\small]
sol = simulate(H, I; tspan=(0.0, 5.0), saveat=0.05)       # original system
sol_h = simulate(H, eta; symmetrize=true)                    # transformed
sol_custom = simulate(H, eta; initial_condition=(H,eta)->(eta*rand(size(H,1))))
\end{Verbatim}

\paragraph{\code{get\_steady\_state(H; eta=I, symmetrize=false)}} Computes a steady state by forward integration.
In the Hermitized frame, combine with \(\eta\) for the transformed steady state as in \code{scripts/steady\_state.jl}.

\begin{Verbatim}[fontsize=\small]
P_ss = get_steady_state(H)
phi_ss = get_steady_state(h; symmetrize=false)  # pass eta via keyword in your code
\end{Verbatim}

\paragraph{\code{infectious\_proportion\_O(n; incl\_disease\_free=false)}} Returns the observable measuring the infected fraction.
Use with \code{observe} on either the original state distribution \code{p} or the transformed state \(\phi\) with metric \(\Omega\).

\begin{Verbatim}[fontsize=\small]
  O = infectious_proportion_O(n; incl_disease_free=false)
  I_mean = observe(O, p)
  Omega = inv(eta*eta')
  I_mean_transformed = observe(eta*O*inv(eta), phi, Omega)
\end{Verbatim}

\paragraph{\code{observe(O, p)} and \code{observe(O, phi, Omega)}} Two observation modes: apply \code{O} to the probability vector \code{p}, or to a transformed state \(\phi\) with the appropriate metric \(\Omega\).
See \code{scripts/statistics.jl} for mean/variance time series in both frames.

\paragraph{\code{shannon\_entropy(p)} and \code{renyi\_entropy(p)}} Entropy measures used for the entropy figures and transition curves. Apply directly to state vectors from simulations or steady states.
For details (e.g., numerical safeguards near zeros), see the function docstrings.

\subsection{Workflow}
In practice, the workflow is: build or load \code{H}, obtain \(h,\eta\) via \code{find\_hermitian}, simulate either original or transformed dynamics using \code{simulate}, and compute observables with \code{observe} and entropies with \code{shannon\_entropy}/\code{renyi\_entropy}. For precise signatures and edge cases, consult the code documentation in \code{src/dyson.jl}.

\section{Scripts and outputs}
Each script starts by including \code{dyson-setup.jl}, which activates the project, brings \pkg{Dyson} into scope, seeds RNG for reproducibility, and ensures \code{results/figures/} exists.
Every script is written to reproduce the figures of the paper, using the functionality of \pkg{Dyson}.

\subsection{Matrix structure}
  Script: \code{scripts/H\_figure.jl}
  \begin{itemize}
    \item Builds a color-coded visualization of the SIS generator's sparsity and transition types.
    \item Saves \code{SIS\_H\_heatmap.pdf} to \code{results/figures/}.
  \end{itemize}

  \paragraph{Adjustable parameters (Figure~2B)}
  \begin{itemize}
    \item Color scaling and palette: edit the color ranges in the script for infections/healings/diagonals.
    \item Graph size/structure: modify \code{default\_initialisation()} to change \code{complete\_graph(6)}.
  \end{itemize}

\subsection{Entropy figures (Figure~1)}
  Script: \code{scripts/entropy\_figures.jl}
  \begin{itemize}
    \item \code{generate\_homotopy(eta, sim\_f, agg\_f; steps)} creates a homotopy \(\eta(\alpha)=(1-\alpha)I+\alpha\,\eta\) and aggregates an observable (by default R\'enyi entropy).
    \item \code{create\_heatmap} saves \code{<name>\_entropy\_heatmap.pdf} and \code{create\_entropies\_figure} saves \code{<name>\_entropy\_graphs.pdf}.
  \end{itemize}

  \paragraph{Adjustable parameters}
  \begin{itemize}
    \item \code{sim\_f}: adjust to perform a homotopy over other systems than the SIS one.
    \item \code{agg\_f}: change to aggregate other observables, e.g., \code{mean\_O} or \code{var\_O}.
    \item \code{create\_entropies\_figure}: modify to plot other statistics or change figure layout.
    \item \code{create\_heatmap}: change the look of the figure, e.g., colormap or axis limits.
  \end{itemize}

\subsection{Statistics over time (Figure~3)}
  Script: \code{scripts/statistics.jl}
  \begin{itemize}
    \item Computes mean infection proportion and variance over time for original and transformed systems.
    \item Saves \code{<name>\_statistics\_full.pdf}.
  \end{itemize}

  \paragraph{Adjustable parameters}
  \begin{itemize}
    \item \code{tspan}, \code{saveat}: change integration horizon and sampling density in \code{simulate} calls.
    \item \code{symmetrize}: switch between raw and symmetrized dynamics in the transformed system.
    \item Observable: replace \code{infectious\_proportion\_O} with other operators to track different quantities.
  \end{itemize}

\subsection{Transition curves (Figure~4)}
  Script: \code{scripts/transition.jl}
  \begin{itemize}
    \item Sweeps ratios \(\beta/\gamma\) and records steady-state means, variances, and entropies (original and transformed).
    \item Saves \code{<name>\_transition.pdf} and a JLD2 data file under \code{results/intermediate/}.
  \end{itemize}

  \paragraph{Adjustable parameters}
  \begin{itemize}
    \item \code{beta\_range}: adjust the logarithmic range of \(\beta/\gamma\) values.
    \item \code{kmax}, \code{tol}: control depth/precision of \code{find\_hermitian} during the sweep.
  \end{itemize}

\subsection{Steady-state comparison (Figure~5)}
  Script: \code{scripts/steady\_state.jl}
  \begin{itemize}
    \item Compares steady-states from Dyson-optimized distribution vs. dynamic evolution; annotates KS distance.
    \item Saves multiple files like \code{<name>\_SIS\_ss\_b001.pdf}, \code{...\_b2.pdf}, \code{...\_b200.pdf}.
  \end{itemize}

  \paragraph{Adjustable parameters}
  \begin{itemize}
    \item Rates: set specific \(\beta\) values relative to \(\gamma\) to probe different regimes.
    \item \code{kmax}: number of Hermitization iterations taken from \code{default\_initialisation()}.
    \item Plot aesthetics: toggle legend, change colors, layout, and bar alignment.
  \end{itemize}

\subsection{Algorithm convergence (Figure~6)}
  Script: \code{scripts/convergence.jl}
  \begin{itemize}
    \item Plots the error trajectory from \code{find\_hermitian} in log scale.
    \item Saves \code{<name>\_convergence.pdf}.
  \end{itemize}

  \paragraph{Adjustable parameters}
  \begin{itemize}
    \item \code{kmax}: increase/decrease the number of iterations to control plot length.
    \item Styling: modify line width, colors, and axis labels.
  \end{itemize}

\section{Figure Gallery}
Below we embed the figures produced by the scripts. If a file is missing, a placeholder box appears.

\begin{figure}[h]
  \centering
  \maybeinclude[width=.8\textwidth]{full-6-wodf_entropy_heatmap.pdf}
  \caption{Entropy heatmap (Figure 1).}
  \label{fig:entropy_heatmap}
\end{figure}

\begin{figure}[h]
  \centering
  \maybeinclude[width=.8\textwidth]{full-6-wodf_entropy_graphs.pdf}
  \caption{Entropy time series (Figure 1).}
  \label{fig:entropy_graphs}
\end{figure}

\begin{figure}[h]
  \centering
  \maybeinclude[width=.8\textwidth]{SIS_H_heatmap.pdf}
  \caption{Structure of the SIS generator (Figure 2B).}
  \label{fig:H_heatmap}
\end{figure}

\begin{figure}[h]
  \centering
  \maybeinclude[width=.8\textwidth]{full-6-wodf_statistics_full.pdf}
  \caption{Mean and standard deviation over time (Figure 3).}
  \label{fig:statistics}
\end{figure}

\begin{figure}[h]
  \centering
  % Prefer disease-free variant if available, fall back to without-disease-free
  \IfFileExists{../results/figures/full-6-wdf_transition.pdf}{%
    \includegraphics[width=.8\textwidth]{full-6-wdf_transition.pdf}%
  }{%
    \maybeinclude[width=.8\textwidth]{full-6-wodf_transition.pdf}%
  }
  \caption{Transition curves over \(\beta/\gamma\) (Figure 4).}
  \label{fig:transition}
\end{figure}

\begin{figure}[h]
  \centering
  \maybeinclude[width=.8\textwidth]{full-6-wodf_SIS_ss_b001.pdf}
  \caption{Steady-state comparison at low \(\beta\) (Figure 5A).}
  \label{fig:ss_b001}
\end{figure}

\begin{figure}[h]
  \centering
  \maybeinclude[width=.8\textwidth]{full-6-wodf_SIS_ss_b2.pdf}
  \caption{Steady-state comparison at intermediate \(\beta\) (Figure 5B).}
  \label{fig:ss_b2}
\end{figure}

\begin{figure}[h]
  \centering
  \maybeinclude[width=.8\textwidth]{full-6-wodf_SIS_ss_b200.pdf}
  \caption{Steady-state comparison at high \(\beta\) (Figure 5C).}
  \label{fig:ss_b200}
\end{figure}

\begin{figure}[h]
  \centering
  \maybeinclude[width=.8\textwidth]{full-6-wodf_convergence.pdf}
  \caption{Convergence of the Hermitization algorithm (Figure 6).}
  \label{fig:convergence}
\end{figure}

\section{Produced Figures}
The scripts in \code{scripts/} produce the following files in \code{results/figures/}:
\begin{itemize}
  \item \code{full-6-wodf\_entropy\_heatmap.pdf} --- Entropy heatmap (Figure~\ref{fig:entropy_heatmap}).
  \item \code{full-6-wodf\_entropy\_graphs.pdf} --- Entropy time series (Figure~\ref{fig:entropy_graphs}).
  \item \code{SIS\_H\_heatmap.pdf} --- Structure of the SIS generator (Figure~\ref{fig:H_heatmap}).
  \item \code{full-6-wodf\_statistics\_full.pdf} --- Mean and standard deviation over time (Figure~\ref{fig:statistics}).
  \item \code{full-6-wdf\_transition.pdf} (or \code{full-6-wodf\_transition.pdf} if excluding the disease-free state) --- Transition curves over \(\beta/\gamma\) (Figure~\ref{fig:transition}).
  \item \code{full-6-wodf\_SIS\_ss\_b001.pdf} --- Steady-state bar plot at low \(\beta\) (Figure~\ref{fig:ss_b001}).
  \item \code{full-6-wodf\_SIS\_ss\_b2.pdf} --- Steady-state bar plot at intermediate \(\beta\) (Figure~\ref{fig:ss_b2}).
  \item \code{full-6-wodf\_SIS\_ss\_b200.pdf} --- Steady-state bar plot at high \(\beta\) (Figure~\ref{fig:ss_b200}).
  \item \code{full-6-wodf\_convergence.pdf} --- Convergence of the Hermitization algorithm (Figure~\ref{fig:convergence}).
\end{itemize}

\end{document}
